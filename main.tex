\documentclass[paper=b4,twocolumn,fleqn,fontsize=14pt,jafontscale=0.925,line_length=110mm,head_space=20mm,foot_space=20mm]{jlreq}
\setlength{\columnseprule}{0.4pt}
\pagestyle{empty}
\renewcommand{\labelenumii}{(\arabic{enumii})}
\usepackage{tikz}
\usetikzlibrary{quotes,angles}
\usepackage{amssymb}
\usepackage{ascmac}
\usepackage{fancybox}

\begin{document}
\twocolumn[%
    \begin{center}
        \LARGE\textbf{3月末 中間テスト(数学)}
    \end{center}
    \vspace{5mm}
    \begin{flushright}
        名前\underline{\hspace{15\zw}}
    \end{flushright}
    \vspace{10pt}
    \begin{center}
        \doublebox{
            問題は裏表印刷で1枚,合計2ページ分あります。
            解答時間は30分間です。}
    \end{center}
    \vspace{25pt}
        ]

\begin{enumerate}
    \item 次の式の分母を有理化せよ。(配点20点)
    \vspace{10pt}
    % \begin{table}[hbtp]
    %     \centering
    %     \begin{tabular}{rl}
    %       \hline
    %       長方形の高さ &  説明\\
    %       \hline \hline
    %       1mm  & 支柱 1mmは無意味\rule[0mm]{0mm}{1mm} \\
    %       \hline
    %       5mm  & ちょっと高くなります.\rule[0mm]{0mm}{5mm} \\
    %       \hline
    %       10mm & かなり高いです.\rule[0mm]{0mm}{10mm} \\
    %       \hline
    %       15mm & 高い!!\rule[0mm]{0mm}{15mm} \\
    %       \hline
    %       30mm & 四角形の底を -15mm.\rule[-15mm]{0mm}{30mm} \\
    %       \hline
    %      \end{tabular}
    %    \end{table}

    $
        \displaystyle\frac{2}{\sqrt{5}+\sqrt{3}}
    $
    \vfill

    \item 次の不等式を解け。(配点20点)
    \vspace{10pt}

    $
        \displaystyle |x+2|>4
    $
    \vfill

    \newpage

    \item 図の$\triangle{\mathrm{ABC}}$おいて,$\mathrm{AB}=7$,$\mathrm{AC}=6$,\\$\angle{\mathrm{BAC}}=60^{\circ}$とする。このとき,辺$\mathrm{BC}$の長さを求めよ。(配点20点)
    
    \vspace{10pt}

    \begin{tikzpicture}
        % nodes
        \coordinate[label=180:$\mathrm{A}$] (A) at (0,0) {};
        \coordinate[label=0:$\mathrm{B}$] (B) at (7,0) {};
        \coordinate[label=$\mathrm{C}$] (C) at (3,5.196) {};
        %
        \path[draw] (B) -- (A) -- (C) -- cycle;
        %
        % ここはpathをつくったあと.
        \path pic["$60^{\circ}$",draw,angle radius=6mm,angle eccentricity=2.1] {angle = B--A--C};
    \end{tikzpicture}
    
    \vfill

    \item $90^{\circ} < \theta < 180^{\circ}$で,$\sin \theta = \displaystyle \frac{2}{3}$とする。このとき,$\cos \theta$と$\tan \theta$ を求めよ。(配点20点)
    \vfill
    
    \newpage
    
    \item $a>0$とする。関数$y=x^2-4x+5\,(0\leqq x \leqq a)$について,次の問いに答えよ。(配点5点×4)
    \vspace{10pt}
        \begin{enumerate}
            \setlength{\leftskip}{-10pt}
            \setlength{\labelsep}{5pt}  
            \item $0<a<4$のとき,\textgt{最大値}を求めよ。
            \vfill
            \item $a\geqq 4$のとき,\textgt{最大値}を求めよ。
            \vfill

            \newpage

            \item $0<a<2$のとき,\textgt{最小値}を求めよ。
            \vfill
            \item $a\geqq 2$のとき,\textgt{最小値}を求めよ。
            \vfill
        \end{enumerate}
\end{enumerate}
\end{document}